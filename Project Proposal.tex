% !TEX TS-program = pdflatex
% !TEX encoding = UTF-8 Unicode

% This is a simple template for a LaTeX document using the "article" class.
% See "book", "report", "letter" for other types of document.

\documentclass[10pt]{article} % use larger type; default would be 10pt

\usepackage{blindtext} % set input encoding (not needed with XeLaTeX)

%%% Examples of Article customizations
% These packages are optional, depending whether you want the features they provide.
% See the LaTeX Companion or other references for full information.

%%% PAGE DIMENSIONS
\usepackage{geometry} % to change the page dimensions
\geometry{a4paper} % or letterpaper (US) or a5paper or....
% \geometry{margin=2in} % for example, change the margins to 2 inches all round
% \geometry{landscape} % set up the page for landscape
%   read geometry.pdf for detailed page layout information

\usepackage{graphicx} % support the \includegraphics command and options

% \usepackage[parfill]{parskip} % Activate to begin paragraphs with an empty line rather than an indent

%%% PACKAGES
\usepackage{booktabs} % for much better looking tables
\usepackage{array} % for better arrays (eg matrices) in maths
\usepackage{paralist} % very flexible & customisable lists (eg. enumerate/itemize, etc.)
\usepackage{verbatim} % adds environment for commenting out blocks of text & for better verbatim
\usepackage{subfig} % make it possible to include more than one captioned figure/table in a single float
% These packages are all incorporated in the memoir class to one degree or another...

%%% HEADERS & FOOTERS
\usepackage{fancyhdr} % This should be set AFTER setting up the page geometry
\pagestyle{fancy} % options: empty , plain , fancy
\renewcommand{\headrulewidth}{0pt} % customise the layout...
\lhead{}\chead{}\rhead{}
\lfoot{}\cfoot{\thepage}\rfoot{}

%%% SECTION TITLE APPEARANCE
\usepackage{sectsty}
\allsectionsfont{\sffamily\mdseries\upshape} % (See the fntguide.pdf for font help)
% (This matches ConTeXt defaults)

%%% ToC (table of contents) APPEARANCE
\usepackage[nottoc,notlof,notlot]{tocbibind} % Put the bibliography in the ToC
\usepackage[titles,subfigure]{tocloft} % Alter the style of the Table of Contents
\renewcommand{\cftsecfont}{\rmfamily\mdseries\upshape}
\renewcommand{\cftsecpagefont}{\rmfamily\mdseries\upshape} % No bold!

%%% END Article customizations

%%% The "real" document content comes below...

\title{Project Proposal for Petrol Pricing Prototype}
\author{Michael Spear - 2022}
\date{} % Activate to display a given date or no date (if empty),
         % otherwise the current date is printed 

\begin{document}
\maketitle

\section{Executive Summary}
This document aims to outline a Data Science prototype project which has the goal of proactive  petrol cost minimisation. It proposes to achieve this by building a data pipeline to current petrol station prices, then do an exploratory data analysisof historical prices, before building a few basic cost minimisation models, and then measure the performance of those models versus a baseline of randomly simulated invididuals. The output will be 2 dashboards; one which will allow both operational users to make more informed decisions and another for management to see how much potential benefit is yet to be gained.

\section{Motivation}
Since 2010, Queensland uses on average 1,700 megaliters of petrol per month (\ref{appendix:petrolUsePerMonth}) at a price of between \$1.30 to \$.1.60 per liter  (\ref{appendix:petrolPricePerMonth}), or \$2.5 billion a month. 
In March 2022, the price of petrol passed \$2 per liter  (\ref{appendix:MarchPetrolPricePerMonth}), this inflationary pressure has impacted household living standards (\ref{appendix:CostOfLiving}) and reduced corportate profits (\ref{appendix:TruckingIndustry}). 
If an application was made that could reduce the costs even by a few percent by understanding sources of variations, it could potentially produce quite a signficant amount of cost savings for individuals and organisations.

In an effort to improve market efficienes, the Queensland Government has mandated that all petrol stations report costs within 30 minutes of a price change  (\ref{appendix:QldGovPetrolPriceChangeRule}). This increased level of pricing transparency makes this idea practical and gives a large amount of avaliable data.

\section{Success Criteria and ROI}
The main proposed outcomes of this project and their proposed benefits are:
\begin{enumerate}
\item \textbf{Create a report which shows controllable and uncontrollable petrol costs}. If all petrol prices are uncontrollable and therefore cannot be minimised, a company may wish to move their fleet to electric to minimise exposure or build storage facilities to capture petrol when it is cheap. If there is a high-level of controllability(i.e. certain petrol stations are always cheaper), companies/individiuals may make a series of heurestics to minimse cost or justidy more expensive machine learning algorithms and hardware.
\item \textbf{Create a dashboard which assists drivers minimise petrol cost by showing optimal destinations.} A cost savings of at least 5\% over the baseline result is acheived.
\item \textbf{Create a monitoring dashboard to see how much of the cost minimisation benefit has been captured and how much is still on the table.} This will assist in reaching the 5\% savings over  baseline.
\end{enumerate}
All software and hardware in this prototype is free. 

\section{ Requirements and Constraints}
Functional requirements are those that should be met to ship the project. They should be described in terms of the customer perspective and benefit.
\subsection{What's in-scope and out-of-scope?}
Some problems are too big to solve all at once. Be clear about what's out of scope.

\section{Methodology}

\subsection{Problem Statement}
How will you frame the problem?

\subsection{Data }
What data will you use to train your model? What input data is needed during serving?

\subsection{Datawarehouse Design}
Creating facts and dimension tables based on star schema 

\subsection{Exploratatory Data Analysis and KPI Identification}
Finding KPIs by using exploratory data analysis

\subsection{Model Building}
From the EDA build a simple model using heruistics or simple models

\subsection{Simulation and Measurement Methodology}
Creating a simulation from the data of 10000 people a year buying fuel from random locations, which is cheaper going to the (cheapest in  surrounding suburbs vs model vs average in suburb) x average liters + average weekly budget (based on prior year simulations)

\subsection{Business Intelligence}
An image of a rough dashboard and how it can be used

\section{Implementation}

\subsection{High-level Design}
Start by providing a big-picture view. 

\subsection{Equipment}
Using SQL, Python, API, Azure SQL Server, Power BI


\subsection{Project Plan}
Timeboxing is the main method. Want to complete in 1 week, as this is just to show that I do understand these key ideas.

\appendix

\section{Appendix: Source of Australian Petrol Useage Data}
\label{appendix:petrolUsePerMonth}
See "Australian Petroleum Statistics - Data Extract January 2022" excel file  and "Sales of petroleum products by State/Territory" tab. Sourced from https://www.energy.gov.au/publications/australian-petroleum-statistics-2022

\section{Appendix: Source of Australian Petrol Useage Data}
\label{appendix:petrolPricePerMonth}
Sourced from https://www.accc.gov.au/media-release/australian-petrol-prices-in-2020-21-were-lowest-in-22-years

\section{Appendix: March 22 Petrol Price Data}
\label{appendix:MarchPetrolPricePerMonth}
Sourced from https://www.msn.com/en-au/money/markets/fuel-price-relief-imminent-as-discount-phase-begins-in-parts-of-queensland/ar-AAVlBG0

\section{Appendix: Cost of Living Pressures}
\label{appendix:CostOfLiving}
Sourced from https://www.abc.net.au/news/2022-03-15/petrol-prices-and-cost-of-living-jobs/100908296

\section{Appendix: Business Costs Rising}
\label{appendix:TruckingIndustry}
Sourced from https://www.northqueenslandregister.com.au/story/7658701/birdsville-bucks-the-fuel-price-trend-for-now/?cs=4770

\section{Appendix: Queensland Government rule on Petrol Price Changes}
\label{appendix:QldGovPetrolPriceChangeRule}
Sourced from https://www.fuelpricesqld.com.au/

\end{document}
